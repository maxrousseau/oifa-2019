%%%%%%%%%%%%%%%%%%%%%%%%%%%%%%%%%%%%%%%%%
% a0poster Landscape Poster
% LaTeX Template
% Version 1.0 (22/06/13)
%
% The a0poster class was created by:
% Gerlinde Kettl and Matthias Weiser (tex@kettl.de)
% 
% This template has been downloaded from:
% http://www.LaTeXTemplates.com
%
% License:
% CC BY-NC-SA 3.0 (http://creativecommons.org/licenses/by-nc-sa/3.0/)
%
%%%%%%%%%%%%%%%%%%%%%%%%%%%%%%%%%%%%%%%%%

%----------------------------------------------------------------------------------------
%	PACKAGES AND OTHER DOCUMENT CONFIGURATIONS
%----------------------------------------------------------------------------------------

\documentclass[a0,portrait]{a0poster}

\usepackage{titlesec}
\usepackage{multicol} % This is so we can have multiple columns of text side-by-side
\columnsep=100pt % This is the amount of white space between the columns in the poster
\columnseprule=3pt % This is the thickness of the black line between the columns in the poster

\usepackage[svgnames]{xcolor} % Specify colors by their 'svgnames', for a full list of all colors available see here: http://www.latextemplates.com/svgnames-colors

\usepackage{times} % Use the times font
%\usepackage{palatino} % Uncomment to use the Palatino font

\usepackage{graphicx} % Required for including images
\graphicspath{{figures/}} % Location of the graphics files

\usepackage{booktabs} % Top and bottom rules for table
\usepackage{enumerate}
\usepackage[backend=biber, bibencoding=utf8]{biblatex}
\usepackage[font=small,labelfont=bf]{caption} % Required for specifying captions to tables and figures

\usepackage{amsfonts, amsmath, amsthm, amssymb} % For math fonts, symbols and environments
\usepackage{wrapfig} % Allows wrapping text around tables and figures
\addbibresource{~/Documents/work/oi_facial_deformities_research/oirefs.bib}

\begin{document}

%----------------------------------------------------------------------------------------
%	POSTER HEADER 
%----------------------------------------------------------------------------------------

% The header is divided into three boxes:
% The first is 55% wide and houses the title, subtitle, names and university/organization
% The second is 25% wide and houses contact information
% The third is 19% wide and houses a logo for your university/organization or a photo of you
% The widths of these boxes can be easily edited to accommodate your content as you see fit

\begin{minipage}[b]{0.75\linewidth}
\Huge \color{NavyBlue} 
\textbf{Automated Facial Shape Analysis in Osteogenesis Imperfecta with
Computer Vision}
\color{Black}\\ % Title
\Huge\textit{A Case-Control Study}\\[1cm] % Subtitle
\Large \textbf{Maxime Rousseau DMD III \& Dr. Jean-Marc Retrouvey}\\ % Author(s)
\Large Faculty of Dentistry, McGill Universtity\\ % University/organization
\end{minipage}

\begin{minipage}[b]{0.19\linewidth}
  \vbox{\vspace{-50em}
\hbox{
	\hspace{80em}\includegraphics[width=8cm]{mcgill_logo.png}
	\vspace{5em}
} % Logo or a photo of you, adjust its dimensions here
}
\end{minipage}

%----------------------------------------------------------------------------------------

\begin{multicols}{2} % This is how many columns your poster will be broken into, a poster with many figures may benefit from less columns whereas a text-heavy poster benefits from more

%----------------------------------------------------------------------------------------
%	ABSTRACT
%----------------------------------------------------------------------------------------

\color{FireBrick}
\begin{abstract}
Osteogenesis Imperfecta subjects  present with typical craniofacial
characteristics that have been described from a qualitative aspect in the
literature\cite{rousseau2018osteogenesis}. 
Such findings remain qualitative and are prone to personal biases.
To obtain quantitative data of the craniofacial characteristics of OI patients,
an automated facial annotation program and statistical shape analysis was used.
Our sample consisted of three groups of patients affected by OI (type I, III
and IV) as well as a control group. Statistically significant morphological discrepancies
were found between control and OI subjects. Also some differences between types
were observed.
\end{abstract}

%----------------------------------------------------------------------------------------
%	INTRODUCTION
%----------------------------------------------------------------------------------------

\color{DarkBlue} 
\section{Introduction}

Osteogenesis Imperfecta is a rare genetic disease affecting mainly COL1A1/A2
genes\cite{glorieux2012} resulting in the production of abnormal collagen
type I in 90\% of cases. These alterations have several implications
for the affected individuals four main types of OI have been described
by Sillence\cite{sillence78} and are based on the phenotypes: Types I,
II, III and IV. 
The range of phenotype severity is very large.
Most common are bone fragility, blue sclera, Dentinogenesis Imperfecta and short stature.
Craniofacial alterations are common and malformations increase with severity of OI.
There is no cure for OI but the use of Biphosphonates has decreased the amount of fractures
and improved quality of life.

%----------------------------------------------------------------------------------------
%	Study Design
%----------------------------------------------------------------------------------------

\color{Black} 
\section{Study Design}

A case-control study was conducted on a total of 306 (M:145/F:161) patients. The Individuals
affected by OI  where part of the BBDC 7701 study conducted at Shriners
Hospital in Montreal, Canada. These patients were grouped according to their OI
classification.  The first group consisted of 88 (42/46) OI type I subjects. A second
group was composed of 28 (9/19) OI type III subjects. The third group consisted
of 57 (26/31) OI type IV subjects. The control group consisted of 133 (68/65)
patients (Figure 1). Population testing for sex (Chi-Squared) and age (ANOVA) showed no statistically significant difference between the groups (Table 1).


\begin{minipage}{0.5\linewidth}
  \centering
\includegraphics[width=\linewidth]{population_boxplot.png}
\captionof{figure}{\color{Green} Population samples of the study}
\end{minipage}
\begin{minipage}{0.5\linewidth}
  \centering
\includegraphics[width=\linewidth]{prcs_imgs.png}                                           
        \captionof{figure}{\color{Green} Visualization of the processed image.                  
          The green box represents the detected face and the red dots are the 68 landmarks}
\end{minipage}
      
%----------------------------------------------------------------------------------------
%	MATERIALS AND METHODS
%----------------------------------------------------------------------------------------

\section{Materials and Methods}

Facial (anteroposterior) photographs of the patients faces for this analysis.
These pictures where taken using a Canon D 70 Dental Eye by a single operator.
Images were then automatically processed through face detection and
annotation with 68 landmarks (Figure 2). The computational methodology used was published 
in the form of a python package prior to the study \cite{rousseaupfla}. 

%------------------------------------------------

\section{Statistical Analysis}

The R statistical programming language and the Shapes
package\cite{dryden-shapes} were used to run the analysis.\\
Mean shapes for each groups were computed using
Generalized Procustre Analysis (GPA) which rotates and scales for optimal
superimposition reducing error coming from different head positioning.
This enables us to focus on the morphological features of the subjects.
The mean shapes were then compared using Goodall's F-Test for statistical
significance.
Each patients' landmark distance was computed from its analogous landmark on
the control mean shape using Eucledian geometry. We will refer to this
measurement as mean Euclidian distance (MED).
This method serves purpose of highlighting and locating the differences in
morphology of the OI types.
More traditionnal means of analysis were also used to compare the groups.
Three different facial ratios  were also computed in addition to lower face height (LFH).
Z-score were calculated for each measurement except for the mean shapes.

\setlength{\belowdisplayskip}{0pt} \setlength{\belowdisplayshortskip}{0pt}
\setlength{\abovedisplayskip}{0pt} \setlength{\abovedisplayshortskip}{0pt}

\begin{center}
  \begin{enumerate}[(a)]
\item \[MED=\dfrac{\sum_{n=1}^{68} \sqrt{(b_i^x-l_i^x)^2+(b_i^y-l_i^y)^2}}{68}\]
\item \[R_1=(l_{46}-l_{37})/(l_{17}-l_1)\]
\item \[R_2=(l_{13}-l_5)/(l_{17}-l_1)\]
\item \[R_3=(l_{17}-l_1)/(l_{28}-l_9)\]
\item \[LFH=(l_{34}-l_{9})/(l_{28}-l_{9})\]
\end{enumerate}
\captionof{figure}{\color{Green} Equations of the measurements taken from the samples for our statistical analysis}
\end{center}

The measurements were tested using ANOVA analysis and post-hoc tests were done
with the Bonferonni correction.


%----------------------------------------------------------------------------------------
%	RESULTS 
%----------------------------------------------------------------------------------------

\section{Results} 
Intergroup comparisons of mean shapes using Goodall's F-Test showed statistically significant differences for all groups except between types I and IV.
Results of statistical tests of the measurements are resumed in table 1.\\

\begin{tabular}{|c||c|c|c|c|c|}
\hline
 & Control & OI Type I & OI Type III & OI Type IV & P-Value\\
\hline
\hline
	N (M/F) & $133 (68/65)$ & $88 (42/46)$ & $28 (9/19)$ & $57 (26/31)$ &
	P(Chi-Squared) = $0.32$\\
	Age (years) & $17.7 (7.7)$ & $19.7 (14)$ & $15.4 (8.4)$ & $17.1 (8.4)$ & P(ANOVA) =
	$0.19$\\
	MED & $NA$ & $0.0075^{bc} (0.005)$ & $0.0106^{c} (0.008)$ & $0.0081 (0.006)$ &
	P(ANOVA)  $<0.05$\\
	MED Z-Score & $NA$ & $-0.122^{bc} (0.907)$ & $0.439^{c} (1.156)$ & $-0.027
	(0.995)$ & P(ANOVA) $<0.05$\\
	Ratio 1 & $0.612^{a} (0.026)$ & $0.595 (0.026)$ & $0.604 (0.036)$ & $0.602
	(0.031)$ & P(ANOVA) $<0.05$\\
	Ratio 1 Z-Score & $0.246^{a} (0.901)$ & $-0.323 (0.926)$ & $0.008 (1.242)$ &
	$-0.080 (1.069)$ & P(ANOVA) $<0.05$\\
	Ratio 2 & $0.799^{a} (0.027)$ & $0.784 (0.038)$ & $0.774 (0.030)$ & $0.791
	(0.031)$ & P(ANOVA) $<0.05$\\
	Ratio 2 Z-Score & $0.247^{a} (0.839)$ & $-0.215 (1.175)$ & $-0.537 (0.912)$ &
	$0.021 (0.944)$ & P(ANOVA) $<0.05$\\
	Ratio 3 & $1.26^{abc} (0.075)$ & $1.33^{b} (0.106)$ & $1.40^{c} (0.124)$ &
	$1.33 (0.097)$ & P(ANOVA) $<0.05$\\
	Ratio 3 Z-Score & $-0.444^{abc} (0.726)$ & $0.264^{b} (1.02)$ & $0.909^{c}
	(1.191)$ & $0.181 (0.938)$ & P(ANOVA) $<0.05$\\
	LFH & $0.569 (0.022)$ & $0.571 (0.027)$ & $0.562 (0.035)$ & $0.567 (0.026)$ &
	P(ANOVA) = $0.47$ \\
	LFH Z-Score & $0.0185 (0.871)$ & $0.0818 (1.02)$ & $-0.253 (1.407)$ &
	$-0.045 (1.021)$ & P(ANOVA) = $0.47$\\
\hline
\end{tabular}
\captionof{table}{\color{Green}Results from the measurements and statistical
testing. a: $p<0.05$ in comparison to OI-I b: $p<0.05$ in comparison to OI-III
c: $p<0.05$ in comparison to OI-IV}

\section{Discussion}

The following histogram (Figure 4) illustrates the mean Euclidean distances of the
landmarks of each type when compared to the mean control shape.
For all three groups, the spikes are
consistent and point towards shape discrepancies at the level of the eyes and
temples. Such findings hint at a possible growth issue causing this recurring pattern in
patients affected by OI.\\                     

\includegraphics[width=\linewidth]{histo.png}\captionof{figure}
{\color{Green} Summary of the mean Euclidian Distances from baseline shape per landmark}

The peaks in landmark distance from baseline for the type III group occur
around points 1, 17, 27, and 46. These landmarks represent the lateral edges of
the left eye and eye and the lateral edges of both the right and left temples.
This pattern is present to a lesser extent in type I and IV patients. 
This is consistent with the qualitative claims of triangular face
previously reported in the literature. Facial ratios where also computed to support
this novel methodology.\\
The results from the different facial ratios also support this hypothesis.
Every measurement taking into account the bitemporal width has a statistically significant ANOVA p-value.
The results of the LFH measurements however point out that the vertical aspect of the face
is not affected in OI patient from a morphological point of view.\\


\color{SaddleBrown} % SaddleBrown color for the conclusions to make them stand out
\section{Conclusions}
Reports from the literature of specific facial characteristics such as
triangular face in patients affected by OI are confirmed by our quantitative
analysis.  It further suggests that these manifestations are present  increase
in  severity depending on the type and that  type III group is  more severely
affected.\\
The findings of this study seem to indicate potential issues in the
phenotypical classification initially put forth by Sillence regarding
craniofacial manifestations. The strong similarities between the facial shapes
of type I and type IV patients challenge our current understanding of the
facial manifestations of the disease. Discerning one from the other in terms
facial characteristics may be harder than previously thought.

\color{DarkSlateGray} % Set the color back to DarkSlateGray for the rest of the content

%----------------------------------------------------------------------------------------
%	FORTHCOMING RESEARCH
%----------------------------------------------------------------------------------------

\section{Forthcoming Research}
Our findings gave us new insight as to where to look for discrepancies with
normal patients when it comes to the facial characteristics of patients affected
by Osteogenesis Imperfecta. The next step will be to attempt to create a classifier from
this data to assist us in our diagnosis of the patient. We would also want to extend our analysis to
more dimension (profile, asymmetry, etc.)

%----------------------------------------------------------------------------------------
%	REFERENCES
%----------------------------------------------------------------------------------------

%\nocite{*} % Print all references regardless of whether they were cited in the poster or not
\small
\printbibliography

%----------------------------------------------------------------------------------------
%	ACKNOWLEDGEMENTS
%----------------------------------------------------------------------------------------

%----------------------------------------------------------------------------------------

\end{multicols}
\end{document}
